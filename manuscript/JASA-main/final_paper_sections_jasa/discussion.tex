% -*- Mode: latex -*- %

\section{Discussion and conclusions}
% \label{sec:concl}

We have presented methods and motivation for robust predicitve inference,
seeking protection against distribution shift. Our arguments and perspective
are somewhat different from the typical approach in distributional
robustness~\cite{ DuchiNa21, BlanchetKaMu19,
  SagawaKoHaLi20}, as we wish to maintain validity in prediction.  
A number of future directions and questions remain unanswered. Perhaps
the most glaring is to fully understand the ``right'' level of
robustness. While this is a longstanding problem~\cite{DuchiNa21}, we
present approaches to leverage the available validation data.  Alternatives
might be compare new covariates and test data
$X$ to the available validation data. \citet{TibshiraniBaCaRa19}
suggest an importance-sampling approach for this, reweighting
data based on likelihood ratios, which may sometimes be feasible
but is likely impossible in high-dimensional scenarios. It would be
interesting, for example, to use projections of the data
to match $X$-statistics on new test data, using this to
generate appropriate distributional robustness sets.
We hope that the perspective here inspires renewed consideration of
predictive validity.

  
\section{Disclosure statement:}

The authors report there are no competing interests to declare.

% We described a method for predictive inference under distribution shifts.  We showed theoretically and empirically that the method generally achieves coverage at the nominal level with no real loss of efficiency, even in a distribution shift setting, whereas the standard conformal methodology breaks down.  We also proposed a procedure, used by our method, for estimating the amount of shift from samples.  A few interesting directions for future work are: (i) considering sets of more structured shifts, which might make our method less conservative; (ii) extending our methodology to more complex prediction problems (\ie, moving beyond regression, classification, etc.); and (iii) investigating what types of conditional coverage guarantees are possible in a distribution shift setup.
